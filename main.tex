\documentclass[11pt]{article}

% Essential
\usepackage[USenglish,american]{babel}

% Optional
\usepackage[hidelinks]{hyperref}
\usepackage{amsmath,amsthm,amsfonts,amssymb}
\usepackage{graphicx}
\usepackage{tikz}
\usepackage{multicol}
\usepackage{float}
\usepackage{subfiles}
\usepackage{makeidx}
\usepackage{lipsum}
\usepackage[margin=3.5cm]{geometry}

%\usepackage[printwatermark]{xwatermark}
%\newwatermark[allpages,color=gray!50,angle=45,scale=3,xpos=0,ypos=0]{DRAFT}

% Titleplage
\title{Geschiedenis van de Informatica \\ (B-KUL-G0K34A)}
\date{\today}
\author{Jonas Devlieghere}

\setlength\parindent{0pt}
\graphicspath{{figures/}{../figures/}}
\makeindex

\newtheorem*{question}{Vraag}

\theoremstyle{definition}
\newtheorem*{solution}{Antwoord}


\begin{document}


% Title Page
\maketitle

% Table of Contents Page
\tableofcontents

\newpage

\section{Geschiedenis}
\subfile{subfiles/geschiedenis.tex}

\section{Theorie van Berekenen}
\subfile{subfiles/berekenen.tex}

\section{Artificiële Intelligentie}
\subfile{subfiles/ai.tex}

\section{Geschiedenis van het Internet}
\subfile{subfiles/internet.tex}

\section{Informatica en Samenleving}
\subfile{subfiles/samenleving.tex}

\section{Natuurlijke Taalverwerking}
\subfile{subfiles/taalverwerking.tex}

\section{Scientific computing}
\subfile{subfiles/scientific.tex}

% Appendix
\appendix

% Index
\printindex

% References
\bibliographystyle{plain}
\bibliography{main}
\nocite{*}

\end{document}