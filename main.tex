\documentclass[11pt]{article}
\usepackage{geschiedenis}


%\usepackage[printwatermark]{xwatermark}
%\newwatermark[allpages,color=gray!50,angle=45,scale=3,xpos=0,ypos=0]{DRAFT}

% Titleplage
\title{Geschiedenis van de Informatica \\ (B-KUL-G0K34A)}
\date{\today}

\author{Ben Lefevere, \\ Dieter Castel, \\ Jonas Devlieghere, \\ Michiel Meersmans, \\ Tim Van den Eynde \\ \& Stefan Pante }

\setlist[description]{style=nextline}
\setlength\parindent{0pt}
\graphicspath{{figures/}{../figures/}}
\makeindex
\newtheorem*{question}{Vraag}

\theoremstyle{definition}
\newtheorem*{solution}{Antwoord}


\begin{document}


% Title Page
\maketitle
\newpage

% Table of Contents Page
\tableofcontents

\newpage

\section{Geschiedenis}
\subfile{subfiles/reeksA.tex}
\subfile{subfiles/reeksB.tex}
\subfile{subfiles/reeksC.tex}

\section{Theorie van Berekenen}
\subfile{subfiles/berekenen.tex}

\section{Artifici\"ele Intelligentie}
\subfile{subfiles/ai.tex}

\section{Geschiedenis van het Internet}
\subfile{subfiles/internet.tex}

\section{Informatica en Samenleving}
\subfile{subfiles/samenleving.tex}

\section{Natuurlijke Taalverwerking}
\subfile{subfiles/taalverwerking.tex}

\section{Scientific computing}
\subfile{subfiles/scientific.tex}

% Appendix
\appendix

% Index
\printindex

% References
%\bibliographystyle{plain}
%\bibliography{main}
\nocite{*}

\end{document}
