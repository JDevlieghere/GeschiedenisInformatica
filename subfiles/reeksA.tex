\documentclass[../main.tex]{subfiles}
\begin{document}

\subsection{Reeks A}

\subsubsection{Vraag 1}
\begin{question}(...) fervente aanhangers van de oudere rekenhulpmiddelen beweerd dat een aantal rekenvaardigheden verloren zouden gaan, o.a. het schatten van de grootte van het resultaat. Denk je dat de actuele ontwikkelingen op het gebied van ICT ook een aantal vaardigheden doen verloren gaan?
\end{question}

\subsubsection{Vraag 2}
\begin{question}
Welke basissen van getallensystemen zijn er in de loop van de geschiedenis gebruikt en waar vinden we die nu nog terug?
\end{question}

\begin{solution}
	\begin{description}
			\item[Basis 60]
			\item[Basis 20]
			\item[Basis 10] 
			\item[Basis 2]

	\end{description}	
\end{solution}

\subsubsection{Vraag 3}
\begin{question}
Waarom is nul belangerijk? Ken je de vermoedelijke oorsprong?
\end{question}

\subsubsection{Vraag 4}
\begin{question}
Welke rekenhulpmiddelen heb je zelf gebruikt? Welke ken je? Welke zijn er in onbruik geraakt en waarom?
\end{question}

\subsubsection{Vraag 5}
\begin{question}
Kan je de volgende figuren thuisbrengen? Vertel wat je ervan weet. (Zie figuren foto’s)
\end{question}


\end{document}