\documentclass[../main.tex]{subfiles}
\begin{document}

\section{Geschiedenis} % (fold)
\label{sec:geschiedenis}


\subsubsection{Vraag 1}
\begin{question}(...) fervente aanhangers van de oudere rekenhulpmiddelen beweerd dat een aantal rekenvaardigheden verloren zouden gaan, o.a. het schatten van de grootte van het resultaat. Denk je dat de actuele ontwikkelingen op het gebied van ICT ook een aantal vaardigheden doen verloren gaan?
\end{question}

\begin{solution}
\lipsum[1]
\end{solution}

\subsubsection{Vraag 2}
\begin{question}
Welke basissen van getallensystemen zijn er in de loop van de geschiedenis gebruikt en waar vinden we die nu nog terug?
\end{question}

\subsubsection{Vraag 3}
\begin{question}
Waarom is nul belangerijk? Ken je de vermoedelijke oorsprong?
\end{question}

\subsubsection{Vraag 4}
\begin{question}
Welke rekenhulpmiddelen heb je zelf gebruikt? Welke ken je? Welke zijn er in onbruik geraakt en waarom?
\end{question}

\subsubsection{Vraag 5}
\begin{question}
Kan je de volgende figuren thuisbrengen? Vertel wat je ervan weet. (Zie figuren foto’s)
\end{question}

\subsection{Reeks B}
\subsubsection{Vraag 1}
\begin{question}
Wat is de rol van Duitsland in de ontwikkeling van computers?
\end{question}

\subsubsection{Vraag 2}
\begin{question}
Wat is de rol van Groot Brittannië in de ontwikkeling van computers?
\end{question}

\subsubsection{Vraag 3}
\begin{question}
Wat weet je van de eerste elektronische computers in de Verenigde Staten van Amerika?
\end{question}

\subsubsection{Vraag 4}
\begin{question}
Wat is de rol van Konrad Zuse?
\end{question}

\subsubsection{Vraag 5}
\begin{question}
Wat is de rol van Alan Turing?
\end{question}

\subsubsection{Vraag 6}
\begin{question}
Wat is de rol van Charles Babbage (en Augusta Ada Lovelace)?
\end{question}

\subsubsection{Vraag 7}
\begin{question}
Wat is de rol van Eckert en Mauchly?
\end{question}

\subsubsection{Vraag 8}
\begin{question}
Wat weet je over John Backus en Grace Hopper?
\end{question}


\subsection{Reeks C}
\subsubsection{Vraag 1}
\begin{question}
Bespreek het onderscheid tussen mainframe, minicomputer, personal computer, werkstation, supercomputer.
\end{question}

\subsubsection{Vraag 2}
\begin{question}
Toon met voorbeelden aan dat belangrijke internationale gebeurtenissen een grote rol hebben gespeeld in de ontwikkeling van computers en informatica.
\end{question}

\subsubsection{Vraag 3}
\begin{question}
IBM heeft een belangrijke rol gespeeld in de ontwikkeling van computers en informatica. Kan je belangrijke verwezenlijkingen of mijlpalen aangeven?
\end{question}

\subsubsection{Vraag 4}
\begin{question}
60 jaar geleden: 1954. Kan je je voorstellen wat er toen nog niet was (in het dagelijkse leven), en dat gerealiseerd is door middel van computers, processoren, informatica?
\end{question}

\subsubsection{Vraag 5}
\begin{question}
Kan je een belangrijke gebeurtenis of evolutie op het gebied van de informatica voor elk  decennium vanaf 1940-1949 noemen en toelichten?
\end{question}

\subsubsection{Vraag 6}
\begin{question}
Hoe zou je zelf de geschiedenis van de informatica in periodes indelen? Op grond waarvan?
\end{question}

\subsubsection{Vraag 7}
\begin{question}
Wat weet je over de geschiedenis en de voorlopers van het WWW?
\end{question}

\subsubsection{Vraag 8}
\begin{question}
Bespreek kort de evolutie van besturingssystemen en gebruikersinterfaces.
\end{question}

\subsubsection{Vraag 9}
\begin{question}
Hoe zijn programmeren en software engineering geëvolueerd?
\end{question}

\subsubsection{Vraag 10}
\begin{question}
Schets de evolutie van de verschillende soorten programmeertalen aan de hand van het bijgevoegd (vereenvoudigd) overzicht. (Zie foto's voor overzicht)
\end{question}

\end{document}