\documentclass[../main.tex]{subfiles}
\begin{document}
\subsection{Schermen}
\begin{question}
	Hoe ziet de evolutie van beeldschermen eruit in de geschiedenis? Heeft deze voor bepaalde veranderingen gezorgd?
\end{question}

\begin{solution}
	De eerste beeldschermen waren oscilloscope displays. Deze schermen konden enkel met vectoren werken en dus lijnen trekken van punt tot punt. Ze hadden geen kleur, zelfs geen zwart-wit. Meestal hadden deze schermen een groene kleur.
	De beeldschermen hebben een sterke evolutie gekend van de eerste vector displays naar CRT schermen die met pixels begonnen werken, naar de platte LCD schermen zoals we ze vandaag kennen. Het aantal pixels is ook constant aan het verhogen op een scherm zoals 1080p en nu zelfs 4k. Hierdoor passen de algoritmen om beeld te maken ook veel aan en hebben we meer en meer geheugen nodig om het beeld te tonen op de schermen. Want deze beelden die gemaakt worden door steeds krachtigere GPU's moeten ook opgeslagen worden om te tonen op het scherm.
	Tegenwoordig zijn er al 24 bits nodig per pixel op het scherm.
\end{solution}

\end{document}