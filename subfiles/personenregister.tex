\documentclass[../main.tex]{subfiles}
\begin{document}
\tabref{table:personen} geeft een overzicht van de personen die vermeld werden tijdens de cursus. De derde kolom geeft aan hoe vaak de persoon voorkomt wat een indicatie geeft van hoe belangrijk deze persoon is.

\begin{longtable}{llll}
\label{table:personen}\\
\toprule
Persoon	& Jaar	& \#	& Beschrijving \\
\midrule
\endhead
\bottomrule
\endfoot
\bottomrule
\endlastfoot
Philippe Dreyfus		&	1962		&	1		&	\\
Jaques Perret			&	1955		&	1		&	\\
Brahmagupta				&	650			&	1		&	\\
Al Khowarizmi			&	750			&	1		&	\\
Leonardo van Pisa		&	1170-1250	&	1		&	Fibonacci	\\
Simon Stevin			&	1548-1620	&	1		&	``Den thienden''	\\
Leonardo da Vinci		&	1452-1519	&	1		&	Ontwerp rekenmachine	\\
Wilhelm Schickard		&	1595-1634	&	1		&	Rekenklok	\\
John Napier				&	1550-1617	&	4		&	Staafjes van Napier, Logaritmen	\\
Blaise Pascal			&	1623-1662	&	2		&	Pascaline 1642	\\
Samuel Morland			&	1625-1695	&	2		&	rekenmachine 1666	\\
Gottfried Leibniz		&	1646-1716	&	2		&	Staffelwalscalculator 1694	\\
Philipp Hahn			&	1739-1790	&	1		&	Serieproductie Staffelwalscalculator	\\
Charles de Colmar		&	1785-1870	&	1		&	Arithmomêtre 1820	\\
Willgodt Odhner			&	1845-1903	&	1		&	Rekenmachine (nokkelwiel) 1878	\\
Frank Baldwin			&	1842-1923	&	1		&	Rekenmachine (nokkelwiel)	\\
Léon Bollé				&	1870-1913	&	1		&	Vermenigvuldiger	\\
Otto Steigner			&	1858-1923	&	1		&	Millionaire 1892	\\
Henri Genaille			&	1890		&	1		&	Rekenstaafjes van Genaille (-> Napier)	\\
Edmund Gunter			&	1581-1626	&	1		&	Rekenliniaal	\\
William Oughted			&	1574-1660	&	1		&	Rekenliniaal	\\
William Thompson		&	1824-1907	&	1		&	Getijdenpredictor 1870	\\
Joseph-Marie Jacquard	&	1752-1834	&	1		&	Weefgetouw van Jacquard	\\
Herman Hollerith		&	1860-1929	&	2		&	Ponskaartenmachines van Hollerith	\\
Charles Babbage			&	1791-1871	&	2		&	Difference Engine, Analytic Engine	\\
Joseph Henry			&	1835		&	1		&	Elektromagetische Relais	\\
Lee De Forest			&	1906		&	1		&	Elektronenbuis	\\
W. Eccles				&	1919		&	1		&	Flip-flop	\\
F. Jordan				&	1919		&	1		&	Flip-flop	\\
Walter Bothe			&	1942		&	1		&	EN logische poort	\\
Charles Wynn-Willimas	&	1931		&	1		&	Binaire teller	\\
George Stibitz			&	1937		&	1		&	Binaire opteller	\\
Kurt Gödel				&	1931		&	1		&	Onvolledigheidsstelling	\\
Alonzo Church			&	1936		&	1		&	Lambdacalculus, onberekenbaarheid	\\
Alan Turing				&	1912-1954	&	5		&	Onberekenbaarheid 1936, 	\\
					 	&				&			&	Automatic Computing Engine \\
Claude Shannon			&	1948		&	1		&	A mathematica theory of 	\\
					 	&				&			&	communication \\
Hugo Koch				&	1920		&	1		&	Engima	\\
Frederic William		&	1948		&	2		&	Manchester Baby Computer + Mark I	\\
Tom Kilburn				&	1948		&	2		&	Manchester Baby Computer + Mark I	\\
Maurice Wilkes			&	1949		&	2		&	EDSAC	\\
Konrad Zuse				&	1940		&	5		&	Z1-Z4 + Plankalkül	\\
John Atansoff			&	1942		&	2		&	ABC Computer	\\
Clifford Berry			&	1942		&	2		&	ABC Computer	\\
Presper Ecker			&	1946		&	3		&	ENIAC+Univac	\\
John Mauchly			&	1946		&	3		&	ENIAC+Univac	\\
Howard Aiken			&	1944		&	2		&	Harvard Mark I Computer	\\
John von Neumann		&	1944		&	1		&	Stored program	\\
John Bardeen			&	1948		&	1		&	Transistor	\\
Walter Brattain			&	1948		&	1		&	Transistor	\\
William Shockley		&	1948		&	1		&	Transistor	\\
An Wang					&	1951		&	1		&	Kerngeheugen	\\
Grace Hopper			&	1952		&	2		&	Automatic Programming \\
					 	&				&			&	(Compilers/Assemblers)	\\
					 	&				&			&	+ COBOL \\
John Backus				&	1957		&	1		&	Fortran	\\
G. Dummer				&	1952		&	1		&	Concept geintegreerd circuit	\\
Jack Kilby				&	1958		&	1		&	Integrated Circuit	\\
Robert Noyce			&	1958		&	1		&	Integrated Circuit	\\
Moore					&	1957		&	1		&	Wet van Moore	\\
Ken Olsen				&	1957		&	1		&	Digital Equipment Corporation (DEC)	\\
Halan Anderson			&	1957		&	1		&	Digital Equipment Corporation (DEC)	\\
Seymour Cray			&	1965		&	1		&	Supercomputers	\\
Dan Bircklin			&	1979		&	1		&	VisiCalc	\\
Bob Frankston			&	1979		&	1		&	VisiCalc	\\
\end{longtable}
\end{document}