\documentclass[../main.tex]{subfiles}
\begin{document}

\subsection{Reeks B}
\subsubsection{Vraag 1}
\begin{question}
Wat is de rol van Duitsland in de ontwikkeling van computers?
\end{question}

\begin{solution}
Tussen 1936 en 1938 bouwde de Duitse ingenieur Konrad Zuse de eerste digitale computer met mechanische schakelaars: de Z1.
Daarna volgden tijdens de tweede wereld oorlog de Z2 en de Z3 ondanks het feit dat de Duitse overheid hiervoor geen subsidies voor wilde geven.
Vertraagd door de oorlog ontwikkelde hij uiteindelijk (1947) de Z4 die werd opgekocht door de ETH Zurich.
We kunnen ook opmerken dat het gebruik van de enigma mechanische codeer machines door duitsland het ontwikkelen van een computer in Groot Brittani\"e in een stroomversnelling heeft gebracht.
Na de oorlog was er te weinig geld en middelen zodat Duitsland verder nog weinig invloed had op de ontwikkeling van de computer.
\end{solution}

\subsubsection{Vraag 2}
\begin{question}
Wat is de rol van Groot Brittanni\"e in de ontwikkeling van computers?
\end{question}
\begin{solution}
In Groot Brittanni\"e werd in 1943 de Colossus I in gebruik genomen. Dit was de eerste elektronische computer met als invoer een snelle ponsbandlezer. 
Daarvoor werden er reeds relais computers ontwikkeld onder de naam Robinson om de cryptografische codes van Duitsland te breken.
De computers werden echter ontworpen in geheimhouding en hebben dus weinig invloed gehad op de verdere ontwikkeling van ander computers.
\end{solution}

\subsubsection{Vraag 3}
\begin{question}
Wat weet je van de eerste elektronische computers in de Verenigde Staten van Amerika?
\end{question}

\subsubsection{Vraag 4}
\begin{question}
Wat is de rol van Konrad Zuse?
\end{question}

\begin{solution}
Hij heeft een enorme rol in het bouwen van de eerste computers (in Europa).
In 1949 richt hij het bedrijf Zuse AG op dat later opging in Siemens AG.

Hij ontwikkelde 4 revolutionaire computers.
\begin{description}
		\item[Z1 (1936-38)] Eerste digitale computer met mechanische schakelaars, programmeerbaar, binaire getallen voorstelling.
		\item[Z2 (1939)] Verbetering van Z1 met elektromagnetische relais in de processor.
		\item[Z3 (1941)] Verbetering op de Z2 waarbij nu ook het geheugen elektromechanisch was.
		\item[Z4 (1947)] Verbetering op de Z3 en de eerste commerci\"ele computer op het Europese vasteland.
\end{description}
\end{solution}

\subsubsection{Vraag 5}
\begin{question}
Wat is de rol van Alan Turing?
\end{question}

\subsubsection{Vraag 6}
\begin{question}
Wat is de rol van Charles Babbage (en Augusta Ada Lovelace)?
\end{question}

\subsubsection{Vraag 7}
\begin{question}
Wat is de rol van Eckert en Mauchly?
\end{question}

\subsubsection{Vraag 8}
\begin{question}
Wat weet je over John Backus en Grace Hopper?
\end{question}


\end{document}
