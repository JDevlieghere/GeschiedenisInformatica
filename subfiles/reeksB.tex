\documentclass[../main.tex]{subfiles}
\begin{document}

\subsection{Reeks B}
\subsubsection{Duitsland}
\begin{question}
Wat is de rol van Duitsland in de ontwikkeling van computers?
\end{question}
\begin{solution}
Tussen 1936 en 1938 bouwde de Duitse ingenieur Konrad Zuse de eerste digitale computer met mechanische schakelaars: de Z1.
Daarna volgden tijdens de tweede wereld oorlog de Z2 en de Z3 ondanks het feit dat de Duitse overheid hiervoor geen subsidies voor wilde geven.
Vertraagd door de oorlog ontwikkelde hij uiteindelijk (1947) de Z4 die werd opgekocht door de ETH Zurich.
We kunnen ook opmerken dat het gebruik van de enigma mechanische codeer machines door Duitsland het ontwikkelen van een computer in Groot Brittani\"e in een stroomversnelling heeft gebracht.
Na de oorlog was er te weinig geld en middelen zodat Duitsland verder nog weinig invloed had op de ontwikkeling van de computer.
\end{solution}

\subsubsection{Groot Brittanni\"e}
\begin{question}
Wat is de rol van Groot Brittanni\"e in de ontwikkeling van computers?
\end{question}
\begin{solution}
In Groot Brittanni\"e werd in 1943 de Colossus I in gebruik genomen. Dit was de eerste elektronische computer met als invoer een snelle ponsbandlezer. 
Daarvoor werden er reeds relais computers ontwikkeld onder de naam Robinson om de cryptografische codes van Duitsland te breken.
De computers werden echter ontworpen in geheimhouding en hebben dus weinig invloed gehad op de verdere ontwikkeling van ander computers.

Alan Turing ontwierp wel de Automatic Computing Engine (ACE) voor het National Physical Laboratory. 
Wegens de geheimhouding werd dit te ambitieus bevonden hoewel er toch een (kleinere) Pilot Ace werd vervoledigt in 1950.
Er volgde een tweede implementatie namelijk de MOSAIC in 1952.

Max Newman samen met Frederic Williams cre\"eerden dri stored program computers.
Een prototype de Manchester Baby Machine in 1948 en de Manchseter en Ferranti Mark I in 1951.

Maurice Wilkes  begon in 1946 na een lezing over de ENIAC te hebben gevolgd aan het ontwikkelen van de EDSAC die kwik vertragingslijnen als geheugen gebruikte.
\end{solution}

\subsubsection{USA}
\begin{question} Wat weet je van de eerste elektronische computers in de Verenigde Staten van Amerika?  \end{question}
\begin{solution}
		\begin{description}
				\item[ASSC/Mark I (1937-1943)]
						Ontworpen door Howard Aiken, elektromagnetische schakelingen, niet binair, controle via lusvormige ponsband.
				\item[Model I (1939)] Ontworpen door George Stibitz, vermenigvuldigingen en delingen uitvoeren.
				\item[Model II (1943)] Enkel optellen en aftrekken, programeerbaar via ponsband. 
				\item[ENIAC (1946)] Volwaardige elektronische computer, ontworpen door John Eckert en John Mauchly, 18000 elektronenbuizen, moeilijk programmeerbaar.
				\item[EDVAC (1951)] Opvolger ENIAC met kwik vertragingslijn, John von Neumann bestudeerde logische structuur hiervan.
				\item[BINAC (1949)] Binaire computer ontworpen onder het bedrijf (ECC) van Eckert en Muachly.
				\item[UNIVAC (1946-1951)] Invoer met magnetische band, geld-problemen voor ontwikkeling, afgewerkt na overname ECC.
		\end{description}
		In 1951 begon IBM dan met ontwikkeling van computers en werd zeer snel het grootste bedrijf in de branche.
\end{solution}

\subsubsection{Konrad Zuse}
\begin{question}
Wat is de rol van Konrad Zuse?
\end{question}
\begin{solution}
Konrad Zuse was een Duitse ingenieur en heeft een enorme rol in het bouwen van de eerste computers (in Europa).
In 1949 richt hij het bedrijf Zuse AG op dat later opging in Siemens AG.

Hij ontwikkelde 4 revolutionaire computers.
\begin{description}
		\item[Z1 (1936-38)] Eerste digitale computer met mechanische schakelaars, programmeerbaar, binaire getallen voorstelling.
		\item[Z2 (1939)] Verbetering van Z1 met elektromagnetische relais in de processor.
		\item[Z3 (1941)] Verbetering op de Z2 waarbij nu ook het geheugen elektromechanisch was.
		\item[Z4 (1947)] Verbetering op de Z3 en de eerste commerci\"ele computer op het Europese vasteland.
\end{description}
\end{solution}

\subsubsection{Alan Turing}
\begin{question}
Wat is de rol van Alan Turing?
\end{question}
\begin{solution}
Alan Turing (1912-1954) was een geniale engelse wiskundige en wordt als vader van de computerwetenschappen gezien.
Hij toonde (parallell aan Church) aan dat er geen algemene oplossing voor het Enscheidungsproblem is
Verder legde hij de basis voor de computerwetenschappen met zijn theoretische onderzoek waarvan we nu nog steeds de turing machine kennen.
Ook werkte hij mee aan de bouw van de Colossus I en later aan de ACE.
Het breken van de ENIGMA code van de duitsers tijdens WOII wordt aan hem toegeschreven.
In 1952 werd Turing vervolgt voor homosekualiteit en in de nasleep van deze situatie pleegde hij op 41 jarige leeftijd zelfmoord.

\end{solution}


\subsubsection{Charles Babbage}
\begin{question}
Wat is de rol van Charles Babbage (en Augusta Ada Lovelace)?
\end{question}
\begin{solution}
Babbage was een engelse wiskundige die de mechanische ``Differential Engine'' ontwierp.
Dit deed hij uit de frustratie van de veel voorkomende fouten in de logaritmische tabellen die toen gebruikte werden.
In 1822 kon hij een succesvol prototype demonstreren waarna hij financi\"ele steun kreeg van de Engelse regering.
Toch slaagde hij er niet in een totaal werkend systeem te laten bouwen en hij zette het werk in 1833 stop.
Later in 1853 werd er in Zweden wel een werkende ``Differential Engine'' gemaakt. 

Babbage bracht nog een inovatief idee aan namelijk dat van een ``Analytical Engine'' die uiteenlopende bewerkingen zou kunnen maken.
Ver vooruit op zijn tijd ontwierp hij een toestel dat met ponskaarten zou kunnen werken (de machine werd echter nooit gebouwd).
Ada Lovelace schreef voor deze machine het eerste algoritme en wordt daarom als de eerste programmeur aangeduid.
Wel voorspelde hij al in 1864 dat de computer de toekomst van de wetenschap zou bepalen.


\end{solution}

\subsubsection{Eckert en Mauchly}
\begin{question}
Wat is de rol van Eckert en Mauchly?
\end{question}
\begin{solution}
Ze namen de leiding in het ontwikkelen van de ENIAC aan de Moore School of Electrical Engeneering.
Toen ze hun ontwerp wouden patenteren kwamen ze in conflict met het bestuur van Moore School.
Hierna richten ze de Electronic Control Company (ECC) op en ontwikkelden nog twee belangrijke computers: de UNIVAC (UNIVersal Automatic Computer) en de BINAC (BINary Automatic Computer).
Door heel wat technische en fincanciele problemen duurde de ontwikkeling van de UNIVAC zeer lang. 
De BINAC (een kleinere binaire) computer werd eerder afgewerkt als zij project in 1949.
Na de overnamene door Remington Rand in 1950 werd dan eindelijk de UNIVAC afgewerkt in 1951. 
De UNIVAC raakte er bekend.
\end{solution}
\subsubsection{John Backus en Grace Hopper}
\begin{question}
Wat weet je over John Backus en Grace Hopper?
\end{question}
\begin{solution}
John Backus cre\"eerde FORTRAN in 1957 voor de IBM 704.

Grace Hopper ontwierp meerder compliers voor de UNIVAC.
In 1952 de A-0 en in 1953 de A-1 en A-2.
Deze compilers maakten het mogelijk om subroutines te bebruiken en assembleerden zo een programma uit bouwblokken.
Ook ontwierp Hopper COBOL en de B-0 compiler daarvoor tussen 1959 en 1960.
De B-0 compiler complide voor de MATH-MATIC en de FLOW-MATIC.
\end{solution}


\end{document}
