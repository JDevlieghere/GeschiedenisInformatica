\documentclass[../main.tex]{subfiles}
\begin{document}

\subsection{Reeks B}
\subsubsection{Duitsland}
\begin{question}
Wat is de rol van Duitsland in de ontwikkeling van computers?
\end{question}
\begin{solution}
Tussen 1936 en 1938 bouwde de Duitse ingenieur Konrad Zuse de eerste digitale computer met mechanische schakelaars: de Z1.
Daarna volgden tijdens de tweede wereld oorlog de Z2 en de Z3 ondanks het feit dat de Duitse overheid niet bereid was hiervoor subsidies te geven.
Vertraagd door de oorlog ontwikkelde hij uiteindelijk in 1947 de Z4. In 1949 werd Zuse met zijn Z4 voor vijf jaar gehuurd door het Instituut voor Toegepaste Wiskunde in Z\"urich.
De Z4 was toen de enige werkende computer op het vasteland van Europa en is eveneens de eerste computer die ooit verkocht is (aan de ETH Z\"urich).
\\\\
We kunnen ook opmerken dat het gebruik van de enigma mechanische codeer machines door Duitsland het ontwikkelen van een computer in Groot-Brittani\"e in een stroomversnelling heeft gebracht. Na de oorlog was er te weinig geld en middelen zodat Duitsland verder nog weinig invloed had op de ontwikkeling van de computer.
\end{solution}

\subsubsection{Groot-Brittanni\"e}
\begin{question}
Wat is de rol van Groot-Brittanni\"e in de ontwikkeling van computers?
\end{question}
\begin{solution}
Charles Babbage wordt gezien als de \emph{vader van de computer}. Zijn ontwerp voor een mechanische computer vormt de basis van latere, meer complexe modellen. Babbage was met zijn idee\"en ver vooruit op de technische mogelijkheden van
zijn tijd.
\\\\
In Groot-Brittanni\"e werd in 1943 de Colossus I in gebruik genomen. Dit was de eerste elektronische computer met als invoer een snelle ponsbandlezer.
Daarvoor werden er reeds relais computers ontwikkeld onder de naam Robinson om de cryptografische codes van Duitsland te breken.
De computers werden echter ontworpen in geheimhouding en hebben dus geen invloed gehad op de verdere ontwikkeling van ander computers.

Alan Turing ontwierp wel de Automatic Computing Engine (ACE) voor het National Physical Laboratory.
Wegens de geheimhouding werd dit te ambitieus bevonden hoewel er toch een (kleinere) Pilot Ace werd vervolledigd in 1950.
Er volgde een tweede implementatie namelijk de MOSAIC in 1952.

Max Newman samen met Frederic Williams cre\"eerden drie stored program computers.
Een prototype de Manchester Baby Machine in 1948 en de Manchseter en Ferranti Mark I in 1951.

Maurice Wilkes begon in 1946 na een lezing over de ENIAC te hebben gevolgd aan het ontwikkelen van de EDSAC die kwik vertragingslijnen als geheugen gebruikte.
\end{solution}

\subsubsection{USA}
\begin{question} Wat weet je van de eerste elektronische computers in de Verenigde Staten van Amerika? \end{question}
\begin{solution}
	\begin{description}
		\item[ABC (Atanasoff Berry Computer)]
		Machine uitsluitend geschikt voor het oplossen van stelsels lineaire vergelijkingen. Deze werkte binair met elektronenbuizen maar was niet programmeerbaar.
		\item[ASSC/Harvard Mark I (1937-1943)]
		Ontworpen door Howard Aiken (Harvard), gebruikte elektromagnetische schakelingen, niet binair, controle via lusvormige ponsband. Financi\"ele steun van IBM.
		\item[Model I (1939), II (1943), III - VI]
		Ontworpen door George Stibitz (Bell Laboratories), vermenigvuldigingen en delingen uitvoeren. Model II kon enkel optellen en aftrekken, programmeerbaar via ponsband. Daarop volgden nog vier modellen van relaiscomputers. De meest van deze computers werden ingezet voor militaire doeleinden.
		\item[ENIAC (1946)]
		Volwaardige elektronische computer, ontworpen door John Eckert en John Mauchly, 18000 elektronenbuizen, moeilijk programmeerbaar.
		\item[EDVAC (1951)]
		Opvolger ENIAC met kwik vertragingslijn, John von Neumann bestudeerde logische structuur hiervan.
		\item[SSEC (1948)]
		Computer met elektronenbuizen, ontworpen door IBM
		\item[BINAC (1949)]
		Binaire computer ontworpen onder het bedrijf (ECC) van Eckert en Muachly.
		\item[UNIVAC (1946-1951)]
		Invoer met magnetische band, geld-problemen voor ontwikkeling, afgewerkt na overname ECC door Remington Rand in 1950.
	\end{description}

	In 1951 legde IBM zicht toe op de ontwikkeling van computers en werd zeer snel het grootste bedrijf in de branche. Dit is vooral te danken aan het feit dat IBM het ponskaarten systeem bleef gebruiken voor hun nieuwe computers en dat de klanten dit systeem al kenden en al machines hadden hiervoor.
\end{solution}

\subsubsection{Konrad Zuse}
\begin{question}
Wat is de rol van Konrad Zuse?
\end{question}
\begin{solution}
Konrad Zuse was een Duitse ingenieur en heeft een enorme rol gespeeld in het bouwen van de eerste computers (in Europa).
In 1949 richt hij het bedrijf Zuse AG op dat later opging in Siemens AG.

Hij ontwikkelde 4 revolutionaire computers.
\begin{description}
		\item[Z1 (1936-38)] Eerste digitale computer met mechanische schakelaars, programmeerbaar, binaire getallen voorstelling.
		\item[Z2 (1939)] Verbetering van Z1 met elektromagnetische relais in de processor.
		\item[Z3 (1941)] Verbetering op de Z2 waarbij nu ook het geheugen elektromechanisch was.
		\item[Z4 (1947)] Verbetering op de Z3 en de eerste commerci\"ele computer op het Europese vasteland.
\end{description}
De Z4 was toen de enige werkende computer op het vasteland van Europa. Hij werd onder meer gebruikt bij het ontwerpen van stuwdammen. Zijn Z4 is de eerste computer die ooit verkocht is (aan de ETH Zürich).

Hij is bovendien gekend voor \'e\'en van de eerste hogere programmeertalen \textbf{Plankalk\"ul}. Hij ontwikkelde deze taal na de capitulatie van Duitsland, terwijl Zuse zijn gezin in leven hield door berglandschapjes met gemzen te schilderen voor soldaten van de Amerikaanse bezettingsmacht.
\end{solution}

\subsubsection{Alan Turing}
\begin{question}
Wat is de rol van Alan Turing?
\end{question}
\begin{solution}
Alan Turing (1912-1954) was een geniale Engelse wiskundige en wordt als vader van de computerwetenschappen gezien.
Hij toonde (gelijktijdig en onafhankelijk van Alonzo Church) aan dat er geen oplossing bestaat voor het \textbf{Enscheidungsproblem}.
Verder legde hij de basis voor de computerwetenschappen met zijn theoretisch onderzoek waarvan we nu nog steeds de turing machine kennen.
Ook werkte hij mee aan de bouw van de Colossus I en later aan de ACE.
Het breken van de ENIGMA code van de Duitsers tijdens WOII wordt aan hem toegeschreven.
In 1952 werd Turing vervolgt voor homoseksualiteit en in de nasleep van deze situatie pleegde hij op 41 jarige leeftijd zelfmoord. In 2009 werd namens de regering van het Verenigd Koninkrijk postuum excuses aangebonden en in 2013 werd hem gratie verleend wat betekent dat zijn veroordeling uit de boeken werd geschrapt.
\end{solution}


\subsubsection{Charles Babbage}
\begin{question}
Wat is de rol van Charles Babbage (en Augusta Ada Lovelace)?
\end{question}
\begin{solution}
Babbage was een Engelse wiskundige die de mechanische ``Differential Engine'' ontwierp.
Dit deed hij uit de frustratie van de veel voorkomende fouten in de logaritmische tabellen die toen gebruikt werden.
In 1822 kon hij een succesvol prototype demonstreren waarna hij financi\"ele steun kreeg van de Engelse regering.
Toch slaagde hij er niet in een totaal werkend systeem te laten bouwen en hij zette het werk in 1833 stop.
Later in 1853 werd er in Zweden wel een werkende ``Differential Engine'' gemaakt.

Babbage bracht nog een innovatief idee aan namelijk dat van een ``Analytical Engine'' die uiteenlopende bewerkingen zou kunnen maken.
Ver vooruit op zijn tijd ontwierp hij een toestel dat met ponskaarten zou kunnen werken (de machine werd echter nooit gebouwd).
Ada Lovelace schreef voor deze machine het eerste algoritme en wordt daarom als de eerste programmeur aangeduid.
Wel voorspelde hij al in 1864 dat de computer de toekomst van de wetenschap zou bepalen.
\end{solution}

\subsubsection{Eckert en Mauchly}
\begin{question}
Wat is de rol van Eckert en Mauchly?
\end{question}
\begin{solution}
Ze namen de leiding in het ontwikkelen van de ENIAC aan de Moore School of Electrical Engeneering aan de University of Pennsylvania.
Toen ze hun ontwerp wilden patenteren kwamen ze in conflict met het bestuur van de Universiteit.
Hierna richtten ze de Electronic Control Company (ECC) op en ontwikkelden nog twee belangrijke computers: de UNIVAC (UNIVersal Automatic Computer) en de BINAC (BINary Automatic Computer).
Door heel wat technische en fincanci\"ele problemen duurde de ontwikkeling van de UNIVAC zeer lang.
De BINAC (een kleinere binaire) computer werd eerder afgewerkt als zij project in 1949.
Na de overname door Remington Rand in 1950 werd dan eindelijk de UNIVAC afgewerkt in 1951.
De UNIVAC raakte erg bekend dankzij het gebruik ervan op de televisie bij de voorspelling van de resultaten van de presidentsverkiezingen van november 1948, waarin Eisenhouwer uitgesproken won.
\end{solution}
\subsubsection{John Backus en Grace Hopper}
\begin{question}
Wat weet je over John Backus en Grace Hopper?
\end{question}
\begin{solution}
John Backus cre\"eerde FORTRAN in 1957 voor de IBM 704.

Grace Hopper ontwierp meerder compilers voor de UNIVAC.
In 1952 de A-0 en in 1953 de A-1 en A-2.
Deze compilers maakten het mogelijk om subroutines te gebruiken en assembleerden zo een programma uit bouwblokken.
Ook ontwierp Hopper COBOL en de B-0 compiler daarvoor tussen 1959 en 1960.
De B-0 compiler compilede voor de MATH-MATIC en de FLOW-MATIC.
\end{solution}

\end{document}
