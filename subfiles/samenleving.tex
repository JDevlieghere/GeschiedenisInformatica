\documentclass[../main.tex]{subfiles}
\begin{document}

\subsection{Privacy}
\begin{question}
In de zogenaamde ``informatiesamenleving'' lijkt privacy een belangrijk punt van bezorgdheid te zijn: geregeld halen casussen in verband met privacy en (het gebruik van) IT de pers en er werd, met name in Europa, heel wat wetgeving ontwikkeld.
Kies en bespreek enkele opvallende aspecten van de rol die ICT speelt of kan spelen met betrekking tot privacy in (bijvoorbeeld) de volgende contexten: gegevensopslag en verwerking, camerabewaking en beeldverwerking, internet en internetgebruik, ubiquitous computing. Welke persoonlijke, professionele en maatschappelijke keuzes zijn hiermee gemoeid? Wat kan daarin de rol zijn van een professionele informaticus?
\end{question}
\begin{solution}
\begin{description}
	\item[Gegevensopslag en Verwerking] ICT maakt het vergaren, opslaan en verwerken van gegevens zeer goedkoop.
	\item[Camerabewaking en Beeldverwerking]
	\item[Internet en Internetgebruik] Wanneer we ons op internet begeven worden we automatisch \emph{getrackt} aan de hadn van ons IP-adres, cookies, logs, etc.
	\item[Ubiquitous Computing]
\end{description}
\end{solution}

\subsection{Internet als Communicatie- en Informatiekanaal}
\begin{question}
Bespreek kort hoe ``anoniem'' en ``open'' het internet is als communicatie- en informatiekanaal. Wat zijn daarvan een aantal voor- en nadelen? Welke mogelijkheden zie je voor het al dan niet reguleren en controleren van het internet in de min of meer nabije toekomst? En wat zouden daarvan de eventuele voor- en nadelen zijn? Kunnen informatici in deze context specifieke bijdragen leveren?
\end{question}

\subsection{Informaticawetenschappen in het (Secundair) Onderwijs}
\begin{question}
Onlangs werd de term ``informaticawetenschappe'' voorgesteld om te verwijzen naar informatica als wetenschap, en dus niet zozeer als “vaardigheidsdiscipline om met ICT te leren werken”, zoals het momenteel op ruime schaal in ons lager en secundair onderwijs wordt begrepen.
Bespreek kort wat de inhoud van informatica als wetenschap in ons (leerplicht)onderwijs zou kunnen en moeten zijn, en waarom. Vergelijk met de huidige situatie in het Vlaams secundair onderwijs en bespreek benaderingen om die te verbeteren op basis van recente voorstellen in onze buurlanden.
\end{question}

\subsection{Professionele Ethiek voor Informatici}
\begin{question}
Bespreek de inhoud en het belang van een ethische ``code'' voor professionele informatici, zoals bijvoorbeeld voorgesteld door ACM (http://www.acm.org/about/code-of-ethics).
Hoe verhoudt deze code zich tot de belangrijkste maatschappelijke kwesties die we bespraken in de sessie over ICT/informatica en samenleving? Bespreek ook de relevantie (of het gebrek daaraan?) van een dergelijke code naast wetgeving en arbeidscontracten.
\end{question}

\end{document}