\documentclass[../main.tex]{subfiles}
\begin{document}

\subsection{Vraag 1}
\begin{question}
Leg uit hoe de evolutie van de hardware de evolutie van numerieke software heeft beïnvloed. Je kan dit doen specifiek voor numerieke lineaire algebra (stelsels en eigenwaardeproblemen).
Leg uit waarom nieuwe software werd ontwikkeld.
\end{question}

\begin{solution}

\end{solution}

\subsection{Vraag 2}
\begin{question}
Geef drie belangrijke uitdagingen voor toekomstige supercomputers.
\end{question}

\begin{solution}

\end{solution}

\subsection{Vraag 3}
\begin{question}
Energiereductie is een belangrijk aspect geworden bij de ontwikkeling van nieuwe hardware. Leg uit hoe men met minder energie toch snellere chips maakt. Welk zijn de consequenties voor de ontwikkeling van software?
\end{question}

\begin{solution}

\end{solution}

\subsection{Vraag 4}
\begin{question}
Leg uit wat er bedoeld wordt met de Accidental Benchmarker op bijgevoegde slide (benchmarker.pdf)
\end{question}

\begin{solution}

\end{solution}

\subsection{Vraag 5}
\begin{question}
Leg uit waar Scalapack voor staat. Leg de figuren uit op bijgevoegde slide (scalapck.pdf)
\end{question}

\begin{solution}

\end{solution}

\end{document}