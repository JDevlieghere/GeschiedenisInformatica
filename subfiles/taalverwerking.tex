\documentclass[../main.tex]{subfiles}
\begin{document}

\subsection{Evolutie \& Innovaties}
\begin{question}
Schets de evolutie van de natuurlijke taalverwerking over de voorbije decennia aan de hand van belangrijke innovaties. Wat zijn de sturende factoren in deze evolutie? Illustreer met voorbeelden.
Zijn er parallelle evoluties waar te nemen in andere domeinen van de informatica?
\end{question}

\subsection{Automatisch Vertalen}
\begin{question}
Schets de evolutie van het \textbf{automatisch vertalen} van de natuurlijke taal over de voorbije decennia aan de hand van belangrijke innovaties.
Wat zijn de sturende factoren in deze evolutie?
Illustreer met voorbeelden.
\end{question}

\subsection{Automatisch Verstaan}
\begin{question}
Schets de evolutie van het \textbf{automatisch verstaan} van de natuurlijke taal over de voorbije decennia aan de hand van belangrijke innovaties.
Wat zijn de sturende factoren in deze evolutie?
Illustreer met voorbeelden.
\end{question}

\subsection{Multidisciplinair Onderzoeksgebied}
\begin{question}
Leg uit waarom natuurlijke taalverwerking als een multidisciplinair onderzoeksgebied wordt beschouwd.
\end{question}

\subsection{Empirisme}
\begin{question}
Wat is empirisme? Verklaar aan de hand van een voorbeeld uit de natuurlijke taalverwerking.
\end{question}

\begin{solution}
Empirisme is de veronderstelling dat kennis voortkomt uit proefondervindelijke ervaringen.
Vanuit het standpunt van natuurlijke taalverwerking betekent dit dat de focus ligt op data-gedreven modellen die worden gevalideerd aan de hand van \emph{held-out} datasets.
Het schoolvoorbeeld is \emph{IBM Watson}, een computer die in spreektaal gestelde vragen beantwoordt aan de hand van een verzameling (on)gestructureerde informatie.
\end{solution}

\subsection{Noam Chomsky}
\begin{question}
Wat zijn de bijdragen van Noam Chomsky voor de natuurlijke taalverwerking?
Evalueer kritisch deze bijdragen.
\end{question}

\subsection{IBM \& Google}
\begin{question}
Wat is de rol van grote bedrijven zoals IBM of Google in de ontwikkeling van taaltechnologie in het verleden en vandaag?
Welke voordelen hebben deze bedrijven in vergelijking met kennisinstellingen?
\end{question}

\end{document}