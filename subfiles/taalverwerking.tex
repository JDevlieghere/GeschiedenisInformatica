\documentclass[../main.tex]{subfiles}
\begin{document}

\subsection{Vraag 1}
\begin{question}
Schets de evolutie van de natuurlijke taalverwerking over de voorbije decennia aan de hand van belangrijke innovaties. Wat zijn de stuende factoren in deze evolutie? Illustreer met voorbeelden.
Zijn er parallelle evoluties waar te nemen in andere domeinen van de informatica?
\end{question}

\begin{solution}

\subsection{Vraag 2}
\begin{question}
Schets de evolutie van het automatisch vertalen van de natuurlijke taal over de voorbije decennia aan de hand van belangrijke innovaties. Wat zijn de sturende factoren in deze evolutie? Illustreer met voorbeelden.
\end{question}

\begin{solution}

\subsection{Vraag 3}
\begin{question}
Schets de evolutie van het automatisch \textbf{verstaan} van de natuurlijke taal over de voorbije decennia aan de hand van belangrijke innovaties. Wat zijn de sturende factoren in deze evolutie? Illustreer met voorbeelden.
\end{question}

\begin{solution}

\subsection{Vraag 4}
\begin{question}
Leg uit waarom natuurlijke taalverwerking als een multidisciplinair onderzoeksgebied wordt beschouwd.
\end{question}

\begin{solution}

\subsection{Vraag 5}
\begin{question}
Wat is empirisme? Verklaar aan de hand van een voorbeeld uit de natuurlijke taalverwerking.
\end{question}

\begin{solution}

\subsection{Vraag 6}
\begin{question}
Wat zijn de bijdragen van Noam Chomsky voor de natuurlijke taalverwerking? Evalueer kritisch deze bijdragen.
\end{question}

\begin{solution}

\subsection{Vraag 7}
\begin{question}
Wat is de rol van grote bedrijven zoals IBM of Google in de ontwikkeling van taaltechnologie in het verleden en vandaag? Welke voordelen hebben deze bedrijven in vergelijking met kennisinstellingen?
\end{question}

\begin{solution}

\end{solution}
\end{document}