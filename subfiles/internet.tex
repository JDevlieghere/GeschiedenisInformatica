\documentclass[../main.tex]{subfiles}
\begin{document}

\subsection{Vraag 1}
\begin{question}
Geef en bespreek enkele belangrijke mijlpalen in de geschiedenis van het internet.
\end{question}

\begin{solution}
Algemene mijlpalen:
\begin{itemize}
	\item 1961: Eerste onderzoek met ondermeer paper over \textbf{packet switching} door Leonard Kleinrock (MIT).
	\item 1969: Eerste twee knopen van ARPANET: verbinding tussen UCLA en SRI. Al snel wordt het netwerk uitgebreid tot 4 nodes met UCSB en University of Utah.
	\item 1972: Eerste e-mail over ARPANET en eerste publieke demonstratie.
	\item 1974: Samenwerking \textbf{Vint Cerf} en \textbf{Robert Kahn} leidt tot TCP/IP protocol dat NCP vervangt. Vier jaar later in 1978 wordt TCP en IP gesplitst wat het ontstaan van UDP toelaat. In 1982 wordt TCP de als standaard gedeclareerd door de DoD.
	\item 1984: Komst van \textbf{DNS}. Het jaar daarop wordt USC verantwoordelijk voor DNS root beheer en SRI (Stanford) voor NIC registraties.
	\item 1985: Oprichting van het \textbf{NSFnet} (National Science Foundation Network), research netwerk in de Verenigde Staten.
	\item 1989: Definitie WWW door Tim Berners-Lee bij CERN. Eveneens het jaar waarin voor het eerst IP connectiviteit in Europa tot stand komt.
	\item 1990: Begin van het commercieel Internet.
\end{itemize}
Mijlpalen specifiek voor Belgi\"e
\begin{itemize}
	\item 1988: Ontstaan \textbf{.be} TLD.
	\item 1991: Verbinding met het Internet.
	\item 1994: Start commercieel internetdiensten.
\end{itemize}
\end{solution}


\subsection{Vraag 2}
\begin{question}
Velen stellen dat het beheer van het internet doorheen de jaren nogal chaotisch is
verlopen. Geef uw visie daarop.
\end{question}

\begin{solution}
Het klopt dat er verschillende instanties verantwoordelijk zijn voor het beheer van verschillende aspecten. Dit is echter niet onverwacht gezien het gedecentraliseerd en internationaal karakter.
\begin{itemize}
	\item 1975: Operationeel beheer wordt overgedragen aan het huidige \textbf{DISA} (Defense Information Systems Agency).
	\item 1983: Het \textbf{IAB} (Internet Activities Board) wordt opgericht. Zij vormen ondermeer:
	\begin{itemize}
		\item \textbf{IETF}: Internet Engineering Task Force (1986)
		\item \textbf{IRTF}: Internet Research Task Force (1986)
	\end{itemize}
	\item 1989: \textbf{RIPE} (Reseaux IP Europeens) wordt gevormd door de Europese service providers. Het doel is administratieve en technische coördinatie garanderen om pan Europees IP-netwerk te verwezenlijken.
	\item 1990: \textbf{IANA} (Internet Assigned Numbers Authority) beschrijft de functie van het beheer van de \emph{DNS root zone} en \emph{IPv4 pool}. Deze functies worden gesponsord door de Amerikaanse overheid (DARPA).
	\item 1995: Network Solutions (huidige VeriSign) is eerste \textbf{registrar} en krijgt het beheer van de root in handen.
	\item 1998: De \textbf{ICANN} (Internet Corporation for Assigned Names and Numbers) wordt opgericht. Deze VZW beheert niet de inhoud van het internet maar co\"ordineert de toegang ertoe. Door een \textbf{multi-stakeholder model} blijft hun macht echter beperkt. In hetzelfde jaar leidt een overeenkomst tussen ICANN en USC tot de overdracht van IANA naar ICANN. Vanaf 2000 voert ICANN de IANA functies uit.
\end{itemize}
\end{solution}



\subsection{Vraag 2}
\begin{question}
Geef enkele belangrijke stappen bij de invoering van het internet in Belgi\"e.
\end{question}

\begin{solution}
Voor de invoer van .be zijn er twee netwerken in Belgi\"e.
\begin{itemize}
	\item Bitnet/EARN
	\begin{itemize}
		\item Netwerk tussen onderzoeksinstellingen gesponsord door IBM. In Belgi\"e geponsord door de NFWO zodat het gebruik ervan gratis is.
		\item Het computercenter van de KU Leuven is de centrale knoop voor Belgi\"e.
		\item E-mail: elke gebruiker heeft een uniek naam (\texttt{jean@BLEKUL60}).
	\end{itemize}
	\item UUCP (EUNET)
	\begin{itemize}
		\item Unix netwerk, betaalbaar door de kosten te delen.
		\item Netwerk met twee centrale knopen: het departement CW van de  KU Leuven en het \emph{Philips research lab}.
		\item E-mail: vormde pad door netwerk (\texttt{mcsun!prlb2!kulcs!jean}). Dit verschilde al naar gelang de locatie.
	\end{itemize}
\end{itemize}
Het was mogelijk e-mails te sturen tussen deze netwerken via gateways. Het is duidelijk dat dit zeer onhandig was aangezien gebruikers unieke namen vereist waren en gebruikers de topologie van het netwerk diende te kennen. In 1988 wordt initiatief genomen door de beheerders van EARN en EUnet en wordt \textbf{Pierre Verbaeten} aangesteld als verantwoordelijke. De .be-naamservers bevonden zich in de VS en er waren twee MX records (.be en .ac.be) al naargelang het netwerk. IP connectiviteit komt echter zeer traag op gang met commerci\"ele diensten pas vanaf 1994.\\
In 1999 wordt \textbf{DNS.be} opgericht. Sinds 200 zijn ze verantwoordelijk voor het beheer van de Belgische domeinnaam. Sindsdien liggen er geen beperkingen meer op domeinnamen en is de registratie ervan een automatisch proces.
\end{solution}


\end{document}