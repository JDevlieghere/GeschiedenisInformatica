\documentclass[../main.tex]{subfiles}
\begin{document}

\subsection{Vraag 1}
\begin{question}
In de loop van de geschiedenis hebben wetenschappers/ingenieurs in verschillende
contexten gedacht dat alles geconstrueerd/berekend kon worden, en dikwijls werd 
dat ook (later) tegengesproken. Geef daarvan een aantal voorbeelden en geef hun 
historische verbanden in het bijzonder voor de ontwikkelingen i.v.m. functies/algoritmen 
tijdens de 20ste eeuw. Bespreek ook in hoeverre die inzichten gestuurd zijn door 
technologische vernieuwingen.
\end{question}

\begin{solution}
%\paragraph{Meetkunde}
De Griekse wiskundigen hielden zich bezig met meetkunde. Vooral met het aspect
wat men kan construeren aan de hand van bijvoorbeeld een passer en liniaal,
passer en meetstok,... Men was ge\"interesseerd welke rekenkundige operaties
men kon uitvoeren op een getallenlijn.
Met behulp van een passer en liniaal kan men optellen, aftrekken, vermenigvuldigen
en delen. Het is echter onmogelijk om een hoek in drie te delen, een kubus uit
een kubus te construeren met dubbele oppervlakte en de kwadratuur van een cirkel
te construeren. Het duurde echter tot midden de 19e eeuw dat dit bewezen werd. Lange
tijd is dan ook intensief gezocht naar methodes om deze problemen alsnog op te lossen.

%\paragraph{Rekenkunde}
De Griekse wiskundigen hielden zich bezig met het ontwikkelen van manieren (een soort
voorlopers van algoritmen) om getallen uit te rekenen. De rekenkunde focuste zich
op natuurlijke en rationale getallen en probeerde hier operaties op te defini\"eren,
bijvoorbeeld: hoe kan men een vierkantswortel van een natuurlijk getal uitrekenen.

Men kon echter aantonen dat $\sqrt{2}$ noch een geheel noch een rationaal getal
was. Bijgevolg kon men met deze constante geen operaties uitvoeren. De Grieken
concludeerden dat meetkunde -- waar worteltrekking van om het even welke afstand
wel mogelijk is -- krachtiger moest zijn dan rekenkunde.

%\paragraph{Logica}
Logica was een derde poot waar de Grieken zich reeds mee bezighielden. Vanaf de
20e eeuw werd logica onder impuls van de Wiener Kreis populair. De Wiener Kreis
zal logica dan ook als het uitstekende middel om kennis te genereren uit andere kennis:
het zogenaamde \emph{Entscheidungsproblem} van \emph{David Hilbert}.
Men voerde vooral onderzoek naar het machinaal manipuleren van logische formules.
\emph{Kurt G\"odel} toonde aan dat dit probleem onoplosbaar is: de zogenaamde
onvolledigheid-stelling.

%\paragraph{Algoritmen}
Eenzelfde fenomeen zien we bij de ontwikkelingen op gebied van functies en algoritmen.
Rond de jaren '20 en '30 vraagt \emph{Kurt G\"odel} zich af -- onder impuls
van de problemen met logica -- welke functies berekend kunnen worden. G\"odel
definieert de \emph{Primitief berekenbare functies}: een verzameling functies
die intu\"itief berekenbaar zijn. \emph{Wilhelm Ackermann} toont echter een functie
die buiten deze verzameling valt, maar nog steeds berekenbaar is. \emph{Kurt G\"odel}
breidt hierop de definitie uit, maar verliest hiermee de eigenschap dat de functie
overal gedefinieerd moet zijn. De poging faalt dus.

%\paragraph{berkenbaarheid}
Ook \emph{Alonzo Church} hield zich met berekenbaarheid bezig, maar 

%\paragraph{Algemeen}
Samenvattend kan men stellen dat men in de geschiedenis met verschillende hulpmiddelen
heeft proberen te rekenen, maar dat men telkens snel inzag dat bepaalde problemen
inherent minder evident waren op te lossen dan anderen. Altijd heeft men lange
tijd geprobeerd om alsnog een methode te vinden: denk bijvoorbeeld aan de vele
gebruikers op \url{stackoverflow.com} die nog steeds XML proberen te parsen met
reguliere expressies. Het duurde lang alvorens men deze pogingen opgaf en nog
langer alvorens er een theoretisch inzicht kwam waarom deze hulpmiddelen ontoereikend
waren. De technologie heeft bij dit theoretisch inzicht amper een rol gespeeld,
behalve het populariseren van de problemen (in een concrete context).
\end{solution}


\subsection{Vraag 2}
\begin{question}
Schets de oorsprong en de evolutie van computationele complexiteitsleer in de 20ste
eeuw.
\end{question}
\begin{solution}
Al bij het ontstaan van informatica
\end{solution}
\end{document}